\section{Git Basics}
\begin{frame}{Brief Recap}
  \begin{columns}[T]
  
    \begin{column}{0.6\textwidth}
      \begin{itemize}
        \item Clone a repository (\texttt{git clone})
        \item Create and edit files
        \item Stage changes (\texttt{git add})
        \item Commit changes (\texttt{git commit})
        \item Upload changes to the remote repository (\texttt{git push})
        \item Download the latest changes from the repository (\texttt{git pull})
      \end{itemize}
    \end{column}


    \begin{column}{0.4\textwidth}
      \centering
      \includegraphics[width=0.9\linewidth]{trainingmaterials/git-II/repo.png}
    \end{column}
  \end{columns}
\end{frame}

\section{Branching in Git}

\begin{frame}{What is a Branch?}
  \begin{itemize}
    \item A branch is a line of development.
    \item A project can have multiple lines of development in parallel.
    \item Common branches:
      \begin{itemize}
        \item \textbf{Main}: Contains the last stable version.
        \item \textbf{Devel}: Contains the next candidate for a stable version.
        \item \textbf{Feature branches}: Each task/feature is developed in a separate branch.
      \end{itemize}
  \end{itemize}
  
  \centering
  \includegraphics[width=0.5\linewidth]{trainingmaterials/git-II/rpi_upm.branches.png }
\end{frame}

\begin{frame}{Using Branches}
  \begin{itemize}
    \item To create a new branch:
      \begin{itemize}
        \item \texttt{git checkout -b <branch\_name>}
      \end{itemize}
    \item Remember to specify the branch name when pushing:
      \begin{itemize}
        \item \texttt{git push origin <branch\_name>}
      \end{itemize}
    \item To switch to between branch:
      \begin{itemize}
        \item \texttt{git checkout <branch\_name>}
      \end{itemize}
    \item To list all branches:
      \begin{itemize}
        \item \texttt{git branch}
      \end{itemize}
  \end{itemize}
\end{frame}

\begin{frame}{Useful Commands}
  \begin{itemize}
    \item Check the current branch and the status of your working directory:
      \begin{itemize}
        \item \texttt{git status}
      \end{itemize}
      \centering
      \includegraphics[width=1\linewidth]{trainingmaterials/git-II/git_status.png}
  \end{itemize}
\end{frame}

\begin{frame}{Useful Commands II} 
  \begin{itemize}
    \item Check the history of the branch:
      \begin{itemize}
        \item \texttt{git log}
      \end{itemize}
      \centering
      \includegraphics[width=1\linewidth]{trainingmaterials/git-II/git_log.png}
  \end{itemize}
\end{frame}

\section{Merging and Conflict Resolution}

\begin{frame}{Merging Branches}
  \begin{itemize}
    \item When work is finished, merge the changes from your branch into \texttt{devel}.
    \item Procedure: 
        \begin{enumerate}
            \item Switch to \texttt{devel} branch: \texttt{git checkout devel}
            \item Update \texttt{devel} branch: \texttt{git pull origin devel}
            \item Merge your branch into \texttt{devel}: \texttt{git merge <branch\_name>}
      \end{enumerate}
    \item Git will automatically determines how to merge the changes.
  \end{itemize}
\end{frame}

\begin{frame}{Handling Conflicts}
  \begin{itemize}
    \item When two branches have changes in the same part of the same file, Git cannot automatically merge them.
    \item This is called a \textbf{conflict}.
    \item The developer must manually resolve the conflict.
  \end{itemize}
  \centering
  \includegraphics[width=1\linewidth]{trainingmaterials/git-II/conflict.png}
\end{frame}

\begin{frame}{Handling Conflicts II}
  \begin{itemize}
    \item Git marks the conflicting areas in the file with special markers:
  \end{itemize}
  \centering
  \includegraphics[width=1\linewidth]{trainingmaterials/git-II/conflict_code.png}
\end{frame}

\begin{frame}{Handling Conflicts III}
  \begin{itemize}
    \item To resolve conflicts:
      \begin{enumerate}
        \item Install \texttt{meld}.
        \item Run \texttt{git mergetool}.
        \item Inspect the files and resolve conflicts by:
          \begin{itemize}
            \item Using all local changes.
            \item Using all remote changes.
            \item Mixing changes as needed.
          \end{itemize}
        \item Save, commit, and push the resolved changes.
      \end{enumerate}
  \end{itemize}
  \centering
  \includegraphics[width=1\linewidth]{trainingmaterials/git-II/meld.png}
\end{frame}

\begin{frame}{Handling Conflicts IV}
  \begin{itemize}
    \item If you are using VSCode, you can use the built-in merge tool to resolve conflicts:
  \end{itemize}
  \centering
  \includegraphics[width=1\linewidth]{trainingmaterials/git-II/conflict_vscode.png}
\end{frame}

\begin{frame}{Best Practices}
  \begin{itemize}
    \item Use a Git cheatsheet for quick reference: \href{https://ndpsoftware.com/git-cheatsheet.html}{"Git Cheatsheet"}
    \item Create atomic commits that focus on a single change or feature.
    \begin{itemize}
      \item Do not change the whole code in a single commit.
    \end{itemize}
    \item Write meaningful commit messages.
    \begin{itemize}
      \item First line: Brief summary. \href{https://www.conventionalcommits.org}{"Conventional Commits"} e.g.:
        \begin{itemize}
          \item \texttt{feat: add data conversion function}
          \item \texttt{fix: correct sensor initialization}
        \end{itemize}
      \item Following lines: Detailed explanation of changes (if required).
    \end{itemize}
    \item Do not keep your changes locally, push them.
    \item Before pulling or pushing, synchronize your local repository index with the remote using:
      \begin{itemize}
        \item \texttt{git fetch}
      \end{itemize}
    \item Push only tested code to \texttt{devel} and stable versions to \texttt{main}.
  \end{itemize}
\end{frame}

\section{Quick Guide}

\begin{frame}{Working with Feature Branches}
  \begin{itemize}
    \item Clone repository and create a feature branch:
      \begin{itemize}
        \item \texttt{git clone <repository\_url>}
        \item \texttt{git checkout -b <feature\_branch>}
      \end{itemize}
    \item Develop and test your feature.
    \item Commit changes with a message:
      \begin{itemize}
        \item \texttt{git commit -m "Message"}
      \end{itemize}
    \item Fetch and pull the latest changes:
      \begin{itemize}
        \item \texttt{git fetch}
        \item \texttt{git pull origin <feature\_branch>}
      \end{itemize}
    \item Resolve conflicts, if any.
    \item Push your changes:
      \begin{itemize}
        \item \texttt{git push origin <feature\_branch>}
      \end{itemize}
  \end{itemize}
\end{frame}

\begin{frame}{Merging Feature Branch into \texttt{devel}}
  \begin{itemize}
    \item Switch to \texttt{devel} branch:
      \begin{itemize}
        \item \texttt{git checkout devel}
      \end{itemize}
    \item Fetch and pull the latest changes:
      \begin{itemize}
        \item \texttt{git fetch}
        \item \texttt{git pull origin devel}
      \end{itemize}
    \item Run the merge:
      \begin{itemize}
        \item \texttt{git merge <feature\_branch>}
      \end{itemize}
    \item Resolve conflicts, if any
   \end{itemize}
\end{frame}

\begin{frame}{Using Git in VSCode}
  \includegraphics[width=1\linewidth]{trainingmaterials/git-II/vscode_git.png}
\end{frame}
 

